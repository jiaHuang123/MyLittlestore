%%
%% This is file `tikzposter-template.tex',
%% generated with the docstrip utility.
%%
%% The original source files were:
%%
%% tikzposter.dtx  (with options: `tikzposter-template.tex')
%%
%% This is a generated file.
%%
%% Copyright (C) 2014 by Pascal Richter, Elena Botoeva, Richard Barnard, and Dirk Surmann
%%
%% This file may be distributed and/or modified under the
%% conditions of the LaTeX Project Public License, either
%% version 2.0 of this license or (at your option) any later
%% version. The latest version of this license is in:
%%
%% http://www.latex-project.org/lppl.txt
%%
%% and version 2.0 or later is part of all distributions of
%% LaTeX version 2013/12/01 or later.
%%


\documentclass{tikzposter} %Options for format can be included here

\usepackage{todonotes}

\usepackage[tikz]{bclogo}
\usepackage{lipsum}
\usepackage{amsmath}

\usepackage{booktabs}
\usepackage{longtable}
\usepackage[absolute]{textpos}
\usepackage[it]{subfigure}
\usepackage{graphicx}
\usepackage{cmbright}
%\usepackage[default]{cantarell}
%\usepackage{avant}
%\usepackage[math]{iwona}
\usepackage[math]{kurier}
\usepackage[T1]{fontenc}
\usepackage{graphicx}

%% add your packages here
\usepackage{hyperref}
% for random text
\usepackage{lipsum}
\usepackage[english]{babel}
\usepackage[pangram]{blindtext}

\colorlet{backgroundcolor}{blue!10}

 % Title, Author, Institute
\title{San Francisco Crime Classification}
\author{Jia Huang}
\institute{$^1$ Xi'an Shiyou University, China \\
}
%\titlegraphic{logos/tulip-logo.eps}

%Choose Layout
\usetheme{Wave}

%\definebackgroundstyle{samplebackgroundstyle}{
%\draw[inner sep=0pt, line width=0pt, color=red, fill=backgroundcolor!30!black]
%(bottomleft) rectangle (topright);
%}
%
%\colorlet{backgroundcolor}{blue!10}

\begin{document}


\colorlet{blocktitlebgcolor}{blue!23}

 % Title block with title, author, logo, etc.
\maketitle

\begin{columns}
 % FIRST column
\column{0.5}% Width set relative to text width

%%%%%%%%%% -------------------------------------------------------------------- %%%%%%%%%%
 %\block{Main Objectives}{
%  	      	\begin{enumerate}
%  	      	\item Formalise research problem by extending \emph{outlying aspects mining}
%  	      	\item Proposed \emph{GOAM} algorithm is to solve research problem
%  	      	\item Utilise pruning strategies to reduce time complexity
%  	      	\end{enumerate}
%%  	      \end{minipage}
%}
%%%%%%%%%% -------------------------------------------------------------------- %%%%%%%%%%


%%%%%%%%%% -------------------------------------------------------------------- %%%%%%%%%%
\block{Introduction}{
  San Francisco was infamous for housing some of the world's most notorious criminals on 
  the inescapable island of Alcatraz. Today, the city is known more for its tech scene than 
  its criminal past. From Sunset to SOMA, and Marina to Excelsior, this project analyzes 12
   years of crime reports from across all of San Francisco's neighborhoods to create a model 
   that predicts the category of crime that occurred, given time and location.
  
  	\begin{description}
    \item[The Dataset]  is in a tabular form and includes chronological, geographical and
     text data and contains incidents derived from the SFPD Crime Incident Reporting system.
  	
    \item[Data Visualization] is the main method designed in this project. 
    Through data visualization, the data set we will present will finally be visualized and
     the understanding of the data will be more direct.
  
  	\end{description}

    The main task of this project is to make a prediction of the type, time and place of crime
     in San Francisco. In this article, we will give an overview of the whole project from data 
     sets, eigenvalues, feature item selection, modeling and conclusion.
}
%%%%%%%%%% -------------------------------------------------------------------- %%%%%%%%%%


%%%%%%%%%% -------------------------------------------------------------------- %%%%%%%%%%
\block{The Dataset}{
  The data ranges from \emph{1/1/2003 to 5/13/2015} creating a training dataset with nine 
  features and 878,049 samples.
\begin{itemize}
    %\emph{Group Outlying Aspects Mining}

    \item
    \emph{Dates - timestamp of the crime incident}
    \item
    \emph{ Category - category of the crime incident. (This is our target variable.)} and
    \item 
    \emph{  Descript - detailed description of the crime incident}
    \item 
    \emph{  DayOfWeek - the day of the week}
    \item 
    \emph{ PdDistrict - the name of the Police Department District}
    \item 
    \emph{ Resolution - The resolution of the crime incident}
    \item 
    \emph{Address - the approximate street address of the crime incident }
    \item 
    \emph{X - Longitude}
    \item 
    \emph{Y - Latitude}
  \end{itemize}
}
%%%%%%%%%% -------------------------------------------------------------------- %%%%%%%%%%


%%%%%%%%%% -------------------------------------------------------------------- %%%%%%%%%%

%\note{Note with default behavior}

%\note[targetoffsetx=12cm, targetoffsety=-1cm, angle=20, rotate=25]
%{Note \\ offset and rotated}

 % First column - second block


%%%%%%%%%% -------------------------------------------------------------------- %%%%%%%%%%
\block{Feature Item}{
  The dataset contains 2,323 duplicates that are meaningless 
  and should be deleted.We will also use coordinates to calculate the 
  distribution of data points on the map of San Francisco.At the same time, 
  67 incorrect locations were found.
  \\
 \vspace{.5cm}
  \centering
  \begin{tabular}{ c | c  }
    \toprule
   Datas                & datetime64    \\
    \toprule
   Category             & object  \\
   \toprule
   Descript             & object    \\
   \toprule
   DayOfWeek            &object     \\
   \toprule
   PdDistrict           & object \\
   \toprule
   Resolution            &object    \\
   \toprule
   Address              & object    \\
   \toprule
   x          & float64    \\
   \toprule
   Y        & float64       \\
    \bottomrule
\end{tabular}
\vspace{.2cm}
  \begin{description}
  
    \item[Dates \& Day of the week]
    These variables are distributed uniformly between 1/1/2003 
    to 5/13/2015 (and Monday to Sunday) and split between the training 
    and the testing dataset as mentioned before. We did not notice any 
    anomalies on these variables.
    The median frequency of incidents is 389 per day with a standard deviation
     of 48.51.
     \\
    \item [Per Week]
    Also, there is no significant deviation of incidents frequency throughout 
    the week. Thus we do not expect this variable to play a significant role 
    in the prediction.
    \\
    
   
  \end{description}
 
  	
%    1) Group Feature Extraction,
%    2) Outlying Degree Scoring, and
%    3) Outlying Aspects Identification.
		
}
%%%%%%%%%% -------------------------------------------------------------------- %%%%%%%%%%


% SECOND column
\column{0.5}
 %Second column with first block's top edge aligned with with previous column's top.

%%%%%%%%%% -------------------------------------------------------------------- %%%%%%%%%%
\block{Feature Item}{
\begin{description}
  \item [Category]
    There are 39 discrete categories that the police department file the 
    incidents with the most common being Larceny/Theft (19.91\%), Non/Criminal
     (10.50\%), and Assault(8.77\%).
     \\
  \item [Police District]
  There are significant differences between the different districts of the 
  City with the Southern district having the most incidents (17.87\%) 
  followed by Mission (13.67\%) and Northern (12.00\%).
  \\
  \item [X - Longitude Y - Latitude]We have tested that the coordinates 
    belong inside the boundaries of the city. Although longitude does not 
    contain any outliers, latitude includes some 90o values which correspond 
    to the North Pole.
    \\
    \item [Address]as a text field, requires advanced techniques to use it for 
    the prediction. Instead in this project, we will use it to extract if the 
    incident has happened on the road or in a building block.
\end{description}

\begin{tikzfigure}%[Overall architecture of \emph{GOAM} algorithm]
    \missingfigure[figcolor=white]{Testing figcolor}
\end{tikzfigure}
}
%%%%%%%%%% -------------------------------------------------------------------- %%%%%%%%%%
% Second column - first block


%%%%%%%%%% -------------------------------------------------------------------- %%%%%%%%%%
\block[titleleft]{Data Visualization}
{
  Based on the Project’s statement, we need to predict the probability
   of each type of crime based on time and location. That being said, 
   we present two diagrams to visualize the importance of these variables. 
   The first one presents the geographic density of 9 random crime
    categories. We can see that although the epicenter of most of the 
    crimes resides on the northeast of the city, each crime has a different 
    density on the rest of the city. This fact is a reliable indication
     that the location ( coordinates / Police District) will be a significant
      factor for the analysis and the forecasting.
\vspace{.5cm}

\vspace{.2cm}
\begin{description}
    \item
    The diagram presents the average number of incidents per
     hour for five of the crimes' categories. 
    It is evident that different crimes have different frequency 
    during different times of the day. Some examples are that 
    prostitution picks during the evening and all through the night, Gambling incidents start late at night until the morning and Burglary picks early in the morning until the afternoon. As before these are sharp pieces of evidence that the time parameters will have a significant role also.
\end{description}


\vspace{.5cm}     
\begin{minipage}{0.5\linewidth}
    \centering
    \begin{tikzfigure}
    \missingfigure[figcolor=white]{Testing figcolor}

    {\small{New Orleans Pelicans on FT\%}}
    \end{tikzfigure}%
\end{minipage}
\hfill
\begin{minipage}{0.5\linewidth}
    \centering
    \begin{tikzfigure}
    \missingfigure[figcolor=white]{Testing figcolor}

    {\small{New Orleans Pelicans on FTA}}
    \end{tikzfigure}%
\end{minipage}
\vspace{.2cm}
}
%%%%%%%%%% -------------------------------------------------------------------- %%%%%%%%%%


% Second column - second block
%%%%%%%%%% -------------------------------------------------------------------- %%%%%%%%%%
\block[titlewidthscale=1, bodywidthscale=1]
{Conclusion}
{
\begin{description}
  \item[Problem Definition]
  To examine the specific problem, we will apply a full Data Science life cycle composed of the six steps.
  \\
  In this paper, we did not complete the design of this project in 
  strict accordance with the whole journey, but through the 
  main data visualization, we understand and analyze the data set provided visually, which is a great progress.
\end{description}
}
%%%%%%%%%% -------------------------------------------------------------------- %%%%%%%%%%


% Bottomblock
%%%%%%%%%% -------------------------------------------------------------------- %%%%%%%%%%
\colorlet{notebgcolor}{blue!20}
\colorlet{notefrcolor}{blue!20}
\note[targetoffsetx=8cm, targetoffsety=-4cm, angle=30, rotate=15,
radius=2cm, width=.26\textwidth]{
Acknowledgement
\begin{itemize}
    \item
    International Cooperation Project (Y7Z0511101)
    of IIE,
    Chinese Academy of Sciences
 \end{itemize}
}

%\note[targetoffsetx=8cm, targetoffsety=-10cm,rotate=0,angle=180,radius=8cm,width=.46\textwidth,innersep=.1cm]{
%Acknowledgement
%}

%\block[titlewidthscale=0.9, bodywidthscale=0.9]
%{Acknowledgement}{
%}
%%%%%%%%%% -------------------------------------------------------------------- %%%%%%%%%%

\end{columns}


%%%%%%%%%% -------------------------------------------------------------------- %%%%%%%%%%
%[titleleft, titleoffsetx=2em, titleoffsety=1em, bodyoffsetx=2em,%
%roundedcorners=10, linewidth=0mm, titlewidthscale=0.7,%
%bodywidthscale=0.9, titlecenter]

%\colorlet{noteframecolor}{blue!20}
\colorlet{notebgcolor}{blue!20}
\colorlet{notefrcolor}{blue!20}
\note[targetoffsetx=-13cm, targetoffsety=-12cm,rotate=0,angle=180,radius=8cm,width=.96\textwidth,innersep=.4cm]
{
\begin{minipage}{0.3\linewidth}
\centering
\includegraphics[width=24cm]{logos/tulip-wordmark.eps}
\end{minipage}
\begin{minipage}{0.7\linewidth}
{ \centering
 The $11^{th}$ International Conference on Knowledge Science,
  Engineering and Management (KSEM 2018),
  17-19/08/2018, Changchun, China
}
\end{minipage}
}
%%%%%%%%%% -------------------------------------------------------------------- %%%%%%%%%%


\end{document}

%\endinput
%%
%% End of file `tikzposter-template.tex'.
